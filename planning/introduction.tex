\Section{Introduction}
Cinnober is a provider of IT solutions to the infrastructure providers of the
global financial industry, including exchanges and clearing houses.  Cinnober
has solutions from price discovery, trading, clearing to settlement of financial
securities. So it is in their interest to try and be on the forefront when it comes to new technologies in finance. While the block chain it self is a relatively new technology an even newer concept is atomic swaps. This is what will be the subject of my study. 

\Subsection{Background}
Although still very much in its early stages, blockchain and distributed
ledger  solutions  are  said  to  be  able  to  transform  the  current  financial  infrastructure, especially the post-trade side.  Most notable of all applications of blockchain technology is Bitcoin, which is an open decentralized network with  an  effectively  immutable  database  of  transactions,  shared by all  full nodes in the network.  The internal currency, bitcoins, is provided to miners who help secure the network by participating in the Proof of Work consensus
algorithm.  Since Bitcoins nascence, other similar solutions have appeared.
Some are modified forks of Bitcoin while some are built from scratch using
the same ideas.

A recent development is the concept of atomic swaps.  Trusting intermediaries always comes with a risk, the third-party in a swap could run off with
the money or maliciously co-operate with the other party.  An atomic swap
gets  around  this  problem  by  using  the  clever  way  that  bitcoin  and  other cryptocurrencies  transactions  can  be  programmed  or  scripted. In 2013 a new method of doing swaps was discovered that allows two parties to swap coins across chains in a trustless way.  A cross-chain transaction refers to two  different  transactions  on  two  different  chains.   In  its  most  basic  form
an atomic swap is a transaction from A to B on chain 1 and from B to A
on chain 2.  The atomicity means that the swap either is fully performed or
nothing happens at all.  The atomicity is assured with so called trap-door
mathematics.

In a payment channel, a set amount of bitcoin is committed for use by
the  senders  and  the  channel  is  created  by  an  opening  transaction.   Once
the transaction has been registered, parties in the channel that have a pos-
itive  balance  can  use  it  to  send  bitcoins  to  the  other  party.   The  channel is strictly bidirectional and with a constant capacity, you can only transfer coins to the owner of the address that is targeted in the payment channel. A payment channel update is only between the two parties involved in the
channel, meaning transactions using the channel don't need to be registered
as transactions in a block in the base layer.  Once any of the parties wishes
to  use  their  coins  somewhere  else,  they  can  publish  a  closing  transaction to the base layer, closing the channel and unlocking the coins attributed to each party.  Usage of payment channels is very convenient if you have two parties that frequently send bitcoin back and forth between each other. A popular example could be a trader sending bitcoin to an exchange, in that case the trader need only keep the balance required to execute their trades on the exchange, having the rest available for transfer on their side in the channel.

An exciting use of payment channels is that they can be linked together
to form a network.  There is an ongoing effort to build one such network,
under  the  name  The  Lightning  Network  (LN).  In  the  LN,  nodes  set  up
payment channels to other nodes and enable routing of payments through
them.  Given a well-connected network and enough capacity, it promises to
provide the best support for everyday payments yet.  There is still a lot of
work to be done before widespread and frequent use of LN is valid, but the
future of networks like LN looks bright.

Atomic swaps and lightning channels can be combined, indeed the mathematical concepts driving the atomic swap is not that different from what makes the lightning network function in the first place. Cross chain lightning network swaps has so far been demonstrated to work. 

\Subsection{Goal}
The  goal  of  this  thesis  is  to  investigate  techniques  for  conducting  settlement of assets on different cryptocurrency chains, evaluating advantages
and drawbacks in different use cases.  

The main purpose of the thesis is to further my own and the worlds knowledge about the new settlement methods that has been discovered recently. The second objective is to become somewhat of an expert in the field of block-chain, lightning network, atomic swaps and settlements. 
