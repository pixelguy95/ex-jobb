\Subsection{Elliptic curve cryptography}
For cryptographic purposes Bitcoin uses a elliptic curve, known simply as elliptic curve 
cryptography or shortend ecc. I will not cover the exact details of the maths, it can 
be found in details in my report, basically all you need to know is that it is an
assymmetric cryptography, meaning it has private and public keys.
 
Though encryption and decryption is possible with ecc it is not the purpose it serves 
for Bitcoin. What is used is signing. This is a concept that all computer scientists
should be familiar with but basically what it means is that you can sign a string
of data with a private key, and that signature could be independently verified by someone who
has the data prior to signing and the public key belonging to the private key, all
without revealing the private key.

Clasically signing is used to prove identity. For example a website could sign
a document to prove that it truly was sent by the server it purports to be.
This is a very useful feature, but in the world of smart contracts the signature
has more significance.

Signatures tend to hold the following significance:

* \textbf{Identity} - A signatures serves as proof that it came from the holder
of the private key. 

* \textbf{Immutability} - A signature makes whatever was signed immutable in a way.
If the data is manipulated after the signing the signature will no longer be valid.

* \textbf{Agreement} - A signature can serve as proof that the holder of a private 
key agreed or approved to whatever data was signed. For example signing a digital
contract could be seen as agreement to the conditions in the contract. The agreement
can't be withdrawn at a later date as the signature proves mathematically that you 
signed it. There is of course the possibility that the private key was stolen. Just
goes to show how important it is to keep the private key private, especially in
Bitocin.

Keep the 3 above points in mind. As they are absolutley vital for the functionality 
of smart contracts.   
