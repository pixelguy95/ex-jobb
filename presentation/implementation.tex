\chapter{Implementation of atomic swaps}
So finally we get to the implementation part of the presentation. Becauce I did
two different types of atomic swaps I have decided to split this part up into two parts.
The first parts cover what is known as on-chain atomic swaps, and the second part
will be the bit more complex off-chain atomic swaps.

\Section{on-chain atomic swap}
THere are several ways to implement a on-chain atomic swap. The one I choose 
is probably the easiest to understand. Before I go into implementation specifics 
it would be good to know how this process actually works. 

Imagine two people, Alice who has Bitcoins and wants litecoins, and Bob who
has litecoins and wants bitcoins. Here a swap is possible. THe methods people
start thinking about right away could be for example some sort of third-party
that ahndles the swap. Or maybe Alice just sends the Bitcin to Bob and hopes
that he doesnt take the money and run. Overall the problem he is trust. 
Even with a third-party there is a chance that Bob and the third-party
are cooperating to steal from Alice. With the help of programmable contracts 
however, we now have a method of swapping where the only thing you have to 
trust is numbers.

The process of an on-chain atomic swap is as follows:
Alice and Bob agrees to do a swap, 1 bitcoin for 10 litecoins. They also 
decide that Alice should be the one to initiate the exchange. 
To start off, Alice generates a random bytestring that will act as a 
pre-image, let's call this R. She hashes this pre-image and produces H_R.  
With te help of H_R she constructs a new swap contract with the following clauses. Pay 1
bitcoin to Bob if he can provide the pre-image of H_R (R). If Bob does not claim 
this output wihin 48 hours, refund the full amount to Alice. 

Alice broadcasts this contract transaction to the bitcoin blockchain and
notifies Bob of doing so, she also sends the unhahsed contract to Bob. 
Bob can then fetch the transaction from the blockchain, he then makes
sure that the contract hash matches the one on the chain (P2SH) and
he validates all the details of the contract. 
