\chapter{Implementation of atomic swaps}
So finally we get to the implementation part of the presentation. Becauce I did
two different types of atomic swaps I have decided to split this part up into two parts.
The first parts cover what is known as on-chain atomic swaps, and the second part
will be the bit more complex off-chain atomic swaps.

\Section{on-chain atomic swap}
THere are several ways to implement a on-chain atomic swap. The one I choose 
is probably the easiest to understand. Before I go into implementation specifics 
it would be good to know how this process actually works. 

Imagine two people, Alice who has Bitcoins and wants litecoins, and Bob who
has litecoins and wants bitcoins. Here a swap is possible. THe methods people
start thinking about right away could be for example some sort of third-party
that ahndles the swap. Or maybe Alice just sends the Bitcin to Bob and hopes
that he doesnt take the money and run. Overall the problem he is trust. 
Even with a third-party there is a chance that Bob and the third-party
are cooperating to steal from Alice. With the help of programmable contracts 
however, we now have a method of swapping where the only thing you have to 
trust is numbers.

The process of an on-chain atomic swap is as follows:
Alice and Bob agrees to do a swap, 1 bitcoin for 10 litecoins. They also 
decide that Alice should be the one to initiate the exchange. 
To start off, Alice generates a random bytestring that will act as a 
pre-image, let's call this $R$. She hashes this pre-image and produces $H_R$.  
With te help of $H_R$ she constructs a new swap contract with the following clauses. Pay 1
bitcoin to Bob if he can provide the pre-image of $H_R$ ($R$). If Bob does not claim 
this output wihin 48 hours, refund the full amount to Alice. 

Alice broadcasts this contract transaction to the bitcoin blockchain and
notifies Bob of doing so, she also sends the unhahsed contract to Bob. 
Bob can then fetch the transaction from the blockchain, he then makes
sure that the contract hash matches the one on the chain (P2SH) and
he validates all the details of the contract. 

If something is wrong here or if Bob changed his mind, he does not have 
to do anything, and Alice can refund her money after 48 hours. 
He can not try to claim Alices bitcoins as he does not know the
pre-image.
If everything in the contract is alright however he constructs a new contract with 
the following details. Alice gets 10 litecoins if she can provide
the pre-image to $H_R$, if she does not claim them within \textbf{24 hours}:
refund the full amount to Bob. Bob broadcasts this transaction to the litecoin
blockchain then notifies Alice and sends over the unhashed contract. 

Alice fetches the contract and validates all the information just like
Bob did. If something is wrong or she has changed her mind she can do nothing,
Bob will get his refund after 24 hours and Alice gets her refund after 48 hours.
Otherwise she can claim Bob's litecoins, she does this by constrcting a claim
transaction containing the pre-image and a signature from her private key. 

Meanwhile Bob has been monitoring the litecoin blockchain and when Alice claims the
litecoins he will see the transaction together the pre-image data. Now when he has
the pre-image he can construct his own claim transaction that claims the bitcoins 
from the initial contract. 

The safety of this swap comes from the reveal mechanism of the pre-image being
the blockchain and the decreasing timelocks. Those of you who have paid close
atention will notice that the mechanisms are very much like the ones used in hashed
timelocked contracts. 

\Subsection{Implementation specifics}
So let's take a closer look at how I implemented this. To work with bitcoin and litecoin you
need a very specific setup. I had one Bitcoin node and one Litecoin node running on my raspberry pi 3
at home. Both running the latest version of their respective core implementations. The nodes were used 
for interacting with the blockchain (broadcasting, reading, monitoring etc...). 
BOth of them ran on the testnet version of the respective blockchains.
This is a global test network where the coins are considered worthless,
but the important parts like blocktime and consensus works exactly the same.
THe testnet exists exactly for the purpose of testing contracts and other bitcoin software
before use in the real world.

My implementation was entirely writen in go-lang. I had never written any code in go
so at the start I spent a day or two just learning how to operate. THe reason 
for choosing go was mostly because of the many great libraries that are available
when programming towards bitcoin and other cryptocurrencies. The most used
library was the btcd-suite. This is the code used for the go-lang specific implementation
of a bitcoin node, it is modularized in such a way that you can basically use it for any
Bitcoin specific task. In my case most of the coding was in regards to constructing
the specialized transactions. 

The on-chain atomic swap implementation actually consists of three small programs.
Two of the programs are from ALices perspective and the other two are from
Bobs perspective. In my little example the bargening and agreement parts are
set in stone, meaning that the exchange rate, timelocks etc are hardcoded.
Both participants are represented as data structures that contains stuff like
their private key and their public key. 

The first program constructs the very first contract transaction that Alice
will broadcast to the bitcoin blockchain. THe btcd libarary provides the
proper datastructures for the transaction related stuff, the contract has 
to be constucted manually however. This is the entire contract script 
for Alice: *Show on screen*

None of the programs actually broadcast the transactions, instead
they print it in hex-format. THe hex-formatted transaction can be broadcast
manually by the user by running a simple command in the Bitcoin-core node.

The other programs are just part of the steps I described earlier. The second
program creates the Bob side of the contract, the contract looks exactly the same but
the timelock as been altered to 24 hours, and of course the payout is in 10 litecoins.
The third and fourth program constructs the claim transactions. 

Overall making this into several programs was pretty pointless. On the bright
side it works just as intended. I tested all possible scenarios and they
all worked just like intended. 

\Section{Off-chain atomic swaps}
How an on-chain atomic swap functions should be pretty clear. The script never does
anyhting I would call super complex. Off-chain atomic swaps however is a bit of
a different story and it can be pretty hard to grasp, escpecially with the
simplified explenations I have given so far. 

An off-chain atomic swap is as you might imagine an atomic swap, but if
it goes as planned it occurs entirrly without any script being executed
on the blockchain. In fact this version of an atomic swap does not use any
specialized scripts at all, instead it uses the clever mechanisms in htlc
contracts to perform it, even across chains. 

As you might remember from payment channels and lightning network 
a multi-hop transactions relies on the htlc contract to be safe. 
THe htlc contracts needs the same hashed pre-image and a strictly
decreasing timelock to be safe. This holds true even in the 
case where a multi-hop transaction travels across chains.
The requirements is that ALice and Bob run specialized 
lightning network software, that runs on both chains. And 
bot of them has to have an open channel on both cryptocurrencies.
Also there needs to be a viable path between eachother
on both networks. 

THis is perhaps best demostrated with a diagram

*Show image and explain*

\Subsection{Off-chain implementation details}
my implememtation of this used most of the same setup as the on-chain swap 
implementation, still used golang, still used the same libraries, still
used the same nodes. A new addition was the lnd library. This is a golang
specific implementation of the BOLT specification. Te BOLT specification
is the official document detailing the protocol and software that drives the 
lightning network, I will come back to this shortly.

My implementaton does not properly connect to the lightning network, doing
it that way would take to much time and it would be very likely to fail for
a newcomer such as me. Instead I implementated my own proper payment channels,
it could be seen as my own private lightning network. I did not use all the proper 
protocol stuff, as I was in controll of both nodes and I was in full control. 

The payment channels were setup using proper contracts as defined by the BOLT 
specification. THis time everything was packed into one neat program, with proper
datastructures for the simulated users and their channels. As before the program 
did not braodcast anything by it self, instead it printed the hex-formated transactions
and the user can optionally braodcast them if they want to try something.

In my tiny network only Alice and Bob existed and they had two channels between them,
one in bitcoin and one in litecoin. I implemented it so that they could make normal
transactions in the payment channels. When I got tht to function properly
I performed my first off-chain atomic swap, using the same method as I showed earlier.
It worked just as intended. I tried all outcomes and it functioned as expected.  
