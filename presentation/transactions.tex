\Subsection{Script \& Transactions}
Alright, now that we gconsensus down let's take a closer look at how programmable contracts
are made possible over the bitcoin network.

Built into Bitcoin there is a programming language called Script. Not the best of names,
but let's not think about that for now. Script is the code that defines exactly what a transaction can and 
cannot do. It is a very barebones language, with byte sized instructions, no heap memory,
and only a very limited stack memory, that strictly operates like a stack. 
Very reminicent of the Forth language. 

A piece of Script code can be defined as either valid or invalid. The only 
way a script could be valid is if at the end of executing the last instruction
there is only one value on the stack, and it equates to true. Anything else
will make the script invalid, infact under certain circumstances the
script could be marked as invalid before the execution is even finnished.

Script has one major limitation compared to other programming languages, it is by design 
not turing complete. Basically meaning
that it cannot do anything. What is missing from the usual programming language repetoire
is loops. Script has no construct that allows for repeated instruction execution. 
Instead all Script programs are completly linear and always execute in a limited time.
The reason for this has to do with the halting problem and a bunch of anti-spam
measures. Even with this limitation Script has a very wide usecase as you will soon see.

Transactions in Bitcoin are not as straight forward as you might expect.
It's not just a matterof moving fu from one attached name to another.
A transaction in Bitdcoin is a datatype consisting of meta data, a list
of inputs and a list of outputs. An input is a reference to a previous
output from an earlier transaction, an output could be seen as a destination,
it holds the amount sent and the ''destination''. 

THe inputs and outputs are where Script comes into the picture.
A output contains a partial script. A input contains the 
complementatry part forming the enitre script.


