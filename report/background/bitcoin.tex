\Section{Bitcoin: a peer to peer electronic cash}
As described in the title of the original white paper, Bitcoin is a peer to peer electronic cash system, which does not rely on any centralized third-party to verify the validity of any transactions or handling of transaction completion.\cite{nakamoto_bitcoin} Instead it is entirely decentralized and trustless. The mechanisms and mathematics that makes this possible are relatively recent discoveries in computer-science. 


%This could be expanded a bit
\Subsection{Network wide consensus}
One of the main problems facing decentralized currencies before Bitcoin was invented is the byzantine generals problem (Byzantine fault)\cite{lamport_shostak_pease_1982}. This refers to independent agents in a system being unable to reach a consensus on what has transpired and what actions to take next. The problem can be imagined as the network being split, where one subsection of the network believes transaction order \textbf{A} is correct while another non-overlapping subsection thinks transaction order \textbf{B} is correct. How can this be resolved, and how will a new node joining the network know which order is the right one?

Bitcoin was the first cryptocurrency to properly solve this problem once and for all, with the help of something called proof-of-work. Proof-of-work originates from the slightly older idea of hashcash.\cite{hashcash}

Hashcash was/is a means to limit email spam and DDoS-attacks by a proof-of-work system. For example a sender could be required to produce a hash of message and nonce with a certain number of bits set to 0 at the start of the hash sequence. This (statistically) should take several attempts to produce. But correctness could be checked in a single step. Thus the hash sent together with the message could be considered proof-of-work. Because there is no known way to produce such a valid hash of a message without trying random combinations of bytes. Thus the only realistic way to have such a hash is if you worked for it.\cite{hashcash} Looking for a valid hash in this kind of proof-of-work systems is often referred to as \textbf{mining}.

Bitcoin used this proof-of-work concept for another purpose however. Rather than combating spam the proof-of-work mechanism is used for reaching consensus. When ever a new block is added to the chain it needs to meet a certain proof-of-work requirement, called difficulty. This difficulty is set so that it should take the combined hashing power of all participants on average 10 minutes to find a valid hash of the next block in the chain.\cite{nakamoto_bitcoin}\cite{antonopoulos_2017}

There is a term called longest proof of work in bitcoin, this refers to the chain of blocks that has the most hash-power supporting it. The chain with the most work done is statistically the one with the most participants. As long as 51\% or more of the participants in the network are honest the longest chain can be trusted. So any new members can accept the chain with most work as the truth.\cite{antonopoulos_2017}

%https://bitcoin.org/en/alert/2013-03-11-chain-fork
%https://gist.github.com/anonymous/3635514
%https://bitcoin.stackexchange.com/questions/3343/what-is-the-longest-blockchain-fork-that-has-been-orphaned-to-date
%https://github.com/bitcoin/bips/blob/master/bip-0050.mediawiki

\Subsubsection{Splits}
Contention for the longest chain can arise if a new block is found in two different parts of the network at (almost) the same time. This is not a problem and will eventually be resolved. If you imagine two blocks (\textbf{A1} and \textbf{B1}) being mined in different parts of the network with the same parent block and half the network got \textbf{A1} first and the other half of the network got \textbf{B1} first. While the entire network will accept all valid blocks they will only mine towards continuing the chain on the block they received first. So if a new block \textbf{A2} with the parent block \textbf{A1} is found first the chain formed by block \textbf{B1} will be considered invalid and the network continues the chain on the \textbf{A} side.\cite{antonopoulos_2017}\\

\InsertBoxR{0}{
	\footnotesize\setlength\fboxsep{10pt}\setlength\fboxrule{1pt}
	\fcolorbox{IndianRed3}{SlateGray1}{\begin{minipage}{2.1in}
			\subsection*{How long was the longest split?}
			The longest splits that occurred by chance were 4 block long and has occurred at least at 3 different occasions. \\\\The longest split ever was caused by an update to the bitcoin core reference implementation (\textbf{0.8.0}) that rejected a block that the other implementations did not reject, the nodes accepting the new block keept building on it while those who had updated built on a different chain. The split lasted for 52 blocks before it was resolved.
	\end{minipage}}
}[7]


Such contention is called a split, or a fork, and happens naturally once every week or so. As explained they will eventually resolve themselves. The split becomes increasingly more unlikely to survive the longer it goes on. To begin with it is unlikely that a new block will form a split in the first place, and for the split to survive another block both new chains will have to receive a new block at almost the same time.\cite{antonopoulos_2017}

So what happens to the transactions that were contained in the invalid chain \textbf{B1} in the example above? As that chain no longer is the longest it is not accepted as valid by the bitcoin network. This might sound scary, but it is not as big of a problem as it sounds. First of splits happens rarely, and two blocks that are mined in parallel often contains more or less the same transactions, as inclusion is based on the fee. Common operation when a block goes stale is to re-add the transactions that were not in the other block into the memory-pool, so transactions are rarely lost. If it is very important that a transaction is not lost under any circumstance you will have to wait for confirmations before accepting it as payment. Confirmations on a transaction is a count of how deep down the transaction is in the blockchain, for example 6 confirmations means that the transaction is 6 blocks deep. The deeper the transaction is; the more likely it is to not be considered invalid in the future.

In common bitcoin lingo there is no difference between a split and a fork. But in this report a \textbf{split will referred to as one occurring unintentionally} and \textbf{a fork being intentional} just so there is no confusion

\Subsubsection{Forks}
Forks happen when there is a disagreement in the bitcoin community when it comes to protocol and consensus. The most famous fork in all of cryptocurrency occurred on \texttt{1st August 2017} and was caused by a conflict regarding the size of blocks. How bitcoin will scale to world-wide use has always been a hot debate in the bitcoin community, the majority seeks to scale bitcoin via second layer solutions such as payment channels, lightning network and side-chains. However there were those who disagreed, and instead wanted the network to handle more transactions by making the blocks larger, which comes at the cot of a more centralized network. 

A change in the blocksize requires the entire network to upgrade to the new protocol rules, but the block size increase was not accepted by enough nodes. So the group decided that a fork was the only way to resolve the issue. So as mentioned on \texttt{1st August 2017} the first block was mined on the new chain.\cite{selena_2017}

The new chain got the name \textbf{Bitcoin cash} (bcash) while the main chain still is called just \textbf{Bitcoin}. Most miners and node owners stayed with Bitcoin, but a fraction jumped ship and started working on bcash instead. Today both chains runs along side each other, co-existing.\cite{selena_2017}

Forks are considered a valid way to vote on what consensus rules and protocol changes should be made. When a change is planned to be made in the chain those who disagree can branch off and follow the old rules. Thus in a way everyone gets their way. You could even branch off on your own forked-chain alone. The main problem for those forking is that most vendors and users still consider the largest (most users, most developers, largest price, etc...) chain to be the valid one. 

\Subsubsection{Soft and hard-forks}\label{soft_hard_fork}
As mentioned forks happens when there is disagreement in the network regarding the consensus rules. When a fork happens where two sub-groups of the entire network essentially follows two different blockchains, that is what is known as a hard-fork. There is another type of fork however called a soft-fork.  Every change in consensus rules are not incompatible with previous rules, in fact an update to the protocol could be made to be backwards compatible with software running on an older protocol.\cite{antonopoulos_2017}

An example of a soft-fork is the Segregated Witness update (see section \ref{segwit}). The changes it brought was still compatible with nodes that had not updated, transactions using segregated witness appears like a ''anyone can spend'' transaction to nodes running the older protocol. Soft-fork is a bit of a misnomer considering that there is no actual fork, as in the blockchain remains unforked.\cite{antonopoulos_2017}

Not all updates can be made as soft-forks, a good example is changes in block size. As any change to that could never be compatible with older nodes.