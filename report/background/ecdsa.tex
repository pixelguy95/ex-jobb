
%https://wstein.org/edu/2007/spring/ent/ent-html/node89.html
\Section{Elliptic-curve cryptography \& ECDSA}\label{ecdsa}
As covered in the introduction, Elliptic-curve cryptography (\textbf{ECC}) and ECDSA is a fundamental building block of bitcoin. Elliptic curve cryptography relies on intractability of calculating the discrete logarithm of a elliptic curve element with respect to a publicly known base point. Or put another way: It is easy to calculate elliptic curve multiplication with multiplicand $n$. But calculating $n$ from the resulting point is considered infeasible with sufficiently large curves and multiplicands.

%Not sure about the field
An elliptic curve is defined by the equation $Y^2=x^3+ax+b$ and six domain parameters $E(p,a,b,G,n,h)$. $\textbf{p}$ is the field that the curve is defined over, this is usually a very large prime number. The curve being defined over a field simply means that the points on the curve fall within $[0, p]$ rather than within the real numbers $\mathbb{R}$. In other words the curve is defined over the field $\mathbb{F}_{p}$. $\textbf{a}$ and $\textbf{b}$ are whatever number you put into the equation. $\textbf{G}$ is the generator point, that is the point on the curve that will be used in point multiplication later. $\textbf{n}$ is the order of G. What that means is that $n$ is the largest number that $G$ can be multiplied by before a point at infinity is produced. $n$ pretty much tells you the limit on how points on the curve that can be generated from $G$. $\textbf{h}$ is the co-factor of the curve. It can be calculated as follows: $h=\frac{1}{n}|(E(\mathbb{F}_{p})|$, where $|(E(\mathbb{F}_{p})|$ is the order/cardinality of the group of points possible on the curve over field $\mathbb{F}_{p}$. $n$ is derived from $G$, $G$ and $p$ should be chosen in such a way that $h \leq 4$, preferably $h=1$.

These domain parameters can be chosen manually or you can use predefined parameters. Elliptic curves that used predefined domain parameters are called named-curves. The named curve used by Bitcoin is called \texttt{Secp256k1}

\Subsection{Secp256k1}
\texttt{Secp256k1} is defined with the following domain parameters (hexadecimal):\\\\
$p=\texttt{FFFFFFFF FFFFFFFF FFFFFFFF FFFFFFFF FFFFFFFF FFFFFFFF FFFFFFFE FFFFFC2F}$\\
or alternatively:\\
$p=2^{256}-2^{32}-2^{9}-2^{8}-2^{7}-2^{6}-2^{4}-1$

$a=0$\\
$b=7$

$G=(\texttt{79BE667E F9DCBBAC 55A06295 CE870B07 029BFCDB 2DCE28D9 59F2815B 16F81798},\\ \null\qquad\:\:\: \texttt{483ADA77 26A3C465 5DA4FBFC 0E1108A8 FD17B448 A6855419 9C47D08F FB10D4B8})$


$n=\texttt{FFFFFFFF FFFFFFFF FFFFFFFF FFFFFFFE BAAEDCE6 AF48A03B BFD25E8C D0364141}$
$h=\texttt{1}$

\Subsection{Math on the elliptic curve}
Two mathematical operations needs to be defined to operate on the elliptic curve: addition and multiplication

\Subsubsection{Point addition}
Let's say you have to distinct points P and Q that both fall on curve $E(p,a,b,G,n,h)$ ($Y^2=x^3+ax+b$). 

$$P+Q=R \Rightarrow (X_P, Y_P) + (X_Q, Y_Q) = (X_R, Y_R)$$

$$X_R = \lambda^2-X_P-X_Q$$
$$Y_R = \lambda(x_P-X_R) -Y_P$$

where $\lambda$:

$$\lambda = \frac{Y_Q-Y_P}{X_Q - X_P} \mod p$$

\Subsubsection{Point multiplication}
If P and Q are coincident, meaning that they have the same coordinates the equation is slightly different. 

$$P+Q=R \Rightarrow P+P=R \Rightarrow 2P=R$$ 

This could be seen as P being multiplied with scalar 2. Most of the equation is the same as with addition, the difference is that:\\
$$\lambda = \frac{(3X^2_P + a)}{(2Y_P)} \mod p$$

\Subsubsection{Faster multiplication with large scalars}
Take $xP=R$ that could be calculated by summing P x times:
$$\sum_{n=1}^{x} P = R$$
This might work fine for smaller numbers but for a very large number, like $x=2^{100}$ it will take infeasible amount of time to calculate. Luckily there is a convenient short cut that you can take called double and add. 

First remember that: $P+P = 2P \Rightarrow 2P + P = 3P \Rightarrow 4P = 2(2P) \Rightarrow 8P = 2(2(2P))$

Lets say $x=200$ in binary terms this could be written as $x=128+64+8$ or $x=2^7+2^6+2^3$ thus $200P=R$ could be written as $$2^7P+2^6P+2^3P=R$$ which could be shorten to: $$2(2(2(2(2(2(2P)))))) + 2(2(2(2(2(2P))))) + 2(2(2P))$$ which looks cumbersome but now instead of 200 calculations you only have to do 18.


\Subsection{Private and public key}
\Subsubsection{Compressed key}
\Subsection{ECDSA}
