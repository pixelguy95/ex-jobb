\Section{Off-chain atomic swaps}
How off-chain atomic swaps are performed are a bit harder to understand and 
required a lot more work, and yet in my opinion they hold more potential for 
the future. The limitation that comes with having more than one node running 
still exists, but the cost that comes with broadcasting transactions on-chain 
is almost completely removed. 

In my experiments I did open proper channels and performed an atomic swap. 
I did not however do it over the real lightning  network, mostly due to time 
constraints and it not being necessary to prove that it worked. Doing it over 
a proper lightning network would require additional tweaks to the protocol and 
would take significantly more time. In my simulation I did not perform any 
multi-hop transactions over nodes not involved in the swap, the fact that it
still would work over multiple hops is evident in the mechanisms of the swap.

\Subsection{Solution proposal}
Doing the swaps on a payment channel that is not part of the lightning network
like I did in my experiments works fine, the problem is that the two participants
have to have channels open that are dedicated to doing swaps between them and nobody
else. This is fine for a couple of implementations, like for exchanges or banks 
etc. It would not be very fitting as a general solution however. 

To make off-chain atomic swaps possible in a more general setting, like the main
lightning network, some changes have to be made to the protocol. 

\Subsubsection{BOLT and c-lightning}
All nodes participating in the lightning network follow a set of rules and 
protocols that are defined in the BOLT specification.\cite{bolt} My first thought were to
simply propose a few changes to the protocol to allow swaps, but there is a 
simpler solution that makes the implementation even more versatile. 

The writers of the BOLT specification has purposely left openings in the
flag and protocol structure to allow for easy extensions.\cite{bolt} This is what is known
as \enquote{it's ok to be odd rule}, what that means is that any  message with
and even number should be supported by all nodes. Messages with odd numbers 
are interpreted as optional. If the node receives a message with an odd number 
that it does not recognize it will simply be ignored (unknown even messages 
should result in closing of the connection as the nodes most likely
have different version numbers).

There are also flags in the BOLT specification that nodes tell each other about
upon connection. A flag is a bit set to 1 or 0 in a bytestring, and they 
represent a supported feature. As before these
flags follow the \enquote{it's ok to be odd rule}. Thus flags in even positions are 
features that must be supported, and flags in odd positions represent optional
features that this node support.

BOLT is designed to support more than bitcoin with the help of the chain\_hash 
variable. The chain hash is the unique identifier for the genesis block on
whatever blockchain you refer to in a specifc protocol message. If a node
receives a chain hash it does not recognize or does not support it will ignore
that message.\cite{bolt}

Another interesting aspect is the new plugin system for the node implementation: 
\textbf{c-lightning}. This plugin system allows for functionality additions that
does not need to be part of the main implementation code. 

\Subsubsection{A non-disruptive solution}
My proposed solution involves making a plugin for c-lightning (and any other 
node as soon as they add plugin support). The plugin adds an additional 
feature flag in any open odd position. The flag represents support and potential
willingness to perform swaps across chains. 

One of the aspects i find most interesting with the off-chain atomic swaps is
the fact that the other nodes (in the case of a multi-hop path) does not need
to know that the transaction they are propagating is part of a swap. In other 
words they do not need to have the plugin themselves to be part of the atomic
swap. Only the two end nodes on each side of the swap, meaning the two 
participants in the swap needs to run special plugin software. 

The plugin's normal operation outside the feature flag addition will be handling 
of forwarding new htlc outputs. When it receives a pre-image that is not related 
to any atomic-swap it will be handled like any normal htlc request would. 
When it sees a htlc related to a known atomic swap however it will handle it as 
a special case.

For now let us assume that the swapping parties reached their agreement through 
a separate channel. The details they need to agree upon are how much to be 
exchanged and at what rate, and also they need to make sure that there is a 
viable path between them on both sides of the network, both to know if the swap 
is possible at all, and to know what timelock to start on. Let us also 
assume that Alice is the one who initiates the exchange.

In a normal transaction Bob would be the one who generates the pre-image used 
in the transaction chain started by Alice. In the case of a swap this would not 
work however, as she will ultimately be the receiver on the other end
when she receives the other type of crypto-currency. Thus before the exchange
begins Alice will generate the pre-image and also start the transaction chain.

When these variables are agreed upon Alice can send the bitcoins to Bob on the
bitcoin side of the network. When the transaction reaches Bob is where the plugin
kicks in. It keeps track of agreed upon swaps and what pre-image hash that they
will use. When the new transaction comes in with the hash they agreed upon earlier
the plugin will act differently than normal operations. It will recognize the hash
and match it to the details of the swap. It can then send the promised money on 
the litecoin side of the network to Alice, using the same pre-image hash and 
continuing on the same timelock. 

As can be seen the plugin is quite simple but still allows for some complex 
behaviour, and it is very non-disruptive. It requires no extra software or special
nodes running on their own network. 

\Subsection{Alternative third-party solutions}
Even in off-chain atomic swaps there is a possibility for a third-party solution, 
though in this case it is not entirely atomic, the risk in the exchanged can be 
minimized as much as the participants want. One thing we are used to in our regular
financial system is that most transactions and exchanges are made in lump sums, meaning 
the whole value of something is transacted in one go. Payment channels and lightning network 
introduces a new method: streaming. In the lightning network there is nothing stopping you 
from splitting up a transaction into smaller parts. This could be seen
as packets in a stream, each packet shifting the value balance between two parties. 

This streaming could change how we think of transactions. As an example imagine a tv-series 
streaming service. Instead of paying per month with money streaming you could be paying per 
watched hour, or even minute or second. You could imagine a parking space that automatically
withdraws money per minute that you are parked in that spot. The future of these money streams 
is a discussion for another day though and for now I will focus on how they could be utilized 
for swaps.

A service could be setup so that anyone wanting to exchange a currency for 
another (for example bitcoin to litecoin) could stream bitcoin into the service and have 
a stream of litecoin coming back. The service itself would not be the one providing 
the litecoins, instead someone on the other end would have the same stream setup
but with the currencies swapped. It could be said that what the service really does is 
connecting streams, like pipes. The users does not even need to know who they are 
connected to, they only need to worry about getting the currency they requested 
at the rate they want.

So what is stopping the service from just closing all channels and running away with the money?
This is where the streams come in handy. Each participant only sends a really tiny amount at
a time, and does not send the next package until thy have received the right amount on the 
channel of the other currency. The service could at most steal the amount used in one 
\enquote{stream} package. The user could adjust the amount sent in each package to 
whatever value they are willing to risk. With a functioning fee system, the service would
be encouraged to not cheat their users and being seen as not trustworthy, the fee could be sent 
with each package, and be proportional to the amount.

Some logistical problems arises when users are using different amounts in their packages. 
If the amounts does not even out the exchange can dead-lock as I have described it so far. 
Let us say one user uses package size of x and another user is trying to exchange for that
currency but is using size x1.5. Then the latter user will not send the next package as 
they have have not received the full promised amount, and the former user will stop sending 
after only receiving x0.5 for their second package. This could be solved by the service 
communicating with the users about the situation and they simply have to adjust accordingly. 
It could also be solved by the service setting a set package size for each currency that
all users have to adhere to. 

So to summarize; a service could be built so that it connects streams of swaps like pipes. 
All said and done this is not an atomic swap, as it is not fully atomic. It is however a 
method of swapping where the risk could be minimized as much as the user wants. It could
also end up being more practical than organizing all swaps in a completely 
distributed manner.