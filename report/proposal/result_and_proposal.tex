\chapter{Results \& Proposal}
In this section I will talk about the results I got when performing the types 
of atomic swaps detailed in previous chapters, and from these and my experience 
I will make a few proposals on how atomic swaps could  be standardized into 
clearly defined protocols and procedures. 

\Section{On-chain atomic swaps}
In my implementation I created a scenario where two pararties swapped bitcoins for litecoins, on their respective testnets. 
The swap was tested for all possible outcomes and possible actions by the actors. And in all cases the swap acted in an atomic way, meaning that either the exchange tock place fully or the state was eventually reset to the original. 
As expected an on-chain atomic swap takes some time to perform. With the values for the timelocks used in my experiment the swap could take up to 48 hours to complete in a worst case scenario. The timelock values could be decreased at the cost of increased risk of failure. 

Under the assumption that the contract transactions, and claim transactions are included in their respective next block an atomic swap can take at best $\sim25$ minutes (Using the average block times for the respective chains, 10 minutes and 2.5 minutes). Both participants have to stay attentive during the whole span (25m - 2880m), in computing terms this is a very long time. 

Under the assumption that Alice (A) is the one who initiated the swap, and Bob (B) is the counter party. A holds the power to control the earliest time that the swap could be completed, before A reveals the pre-image in the blockchain B can do nothing but wait. It could be argued that this is not a swap at all but instead a type of option. If A never reveals the pre-image the swap will be revert back to the pre-swap state. Alice and Bob could agree on a swap at a certain rate, then A could wait it out to see if she could get a better deal (as exchange rates are constantly changing) or decide not to go through with it if agreed upon exchange rate is worse than the current one. A cost for the privilege of this decision could be added to the initial contract transaction as an extra output giving a small sum of money to Bob. 

The swap vs option discussion is interesting, I will however focus more on the technical details in this report, and will leave this topic for future discussion at another time. 

The two sub sections below holds proposals on how wallet and node software could be changed/upgraded to best support atomic swaps in the future. Both runs on the premise that the manual input and interaction from the participants should be kept at minimum possible level and that most of the process should be automated by software and hardware.

\Subsection{Proposed solution}
The easiest way is to modify wallet software so that it can facilitate atomic swaps, not the matching between the swapees, 
that could be handled through other channels. The largest problem is not the creation of the swap it self. But rather
the fact that the blockchian has to be monitored and the pre-image extracted. Most wallet's do not monitor two different blockchains,
but rather only the one it was made to hold currency from. 

There exists wallet's that can handle transactions on multiple blockchains. Something they all have in common is the lack of self-owned
nodes. When the node your wallet is interacting with is owned by someone other than yourself there is risk. The owner of the node and
the other party (they could even be the same person) in the swap could be cooperating to deceive you. 

The only truly risk-free way to handle a cross-chain swap is to run your own nodes on the crypto-currency chains which you wish to swap.
This is a pretty large hassle if you only want to perform a single swap, it is a trade off between invested time and risk. 

For the completely risk-free trustless solution I propose a wallet that is capable of managing cryptocurrencies of different kinds, 
interactions with blockchains happens via RPC connections to the full-node. Under the assumption that both parties are using this
hypothetical wallet making an atomic swap is fairly simple. Either the wallets could communicate and exchange data like addresses 
and contracts automatically, or that data could be exchanged manually via some other type of channel.

The process will still be as described in previous chapters, in this case the wallet will manage all the time-critical tasks and 
monitoring that is difficult or tedious for a human to do.

%More here maybe?

\Subsection{Third-party alternative solution}
If the amount exchanged is small enough, taking a slightly higher risk might be deemed acceptable by the participants. A third-party service could be constructed so that it can handle the time-critical parts of the swap. The contract transactions will be broadcast normally by the participants. If using segregated witness, both the claiming and timelock-refund transactions could be pre-constructed and signed by the participants, then sent to this hypothetical service.

The service could take a small fee for the job, but for now let us imagine it is free. The service monitors all blockchains that it supports swaps for. When it receives a contract and the related spending transactions it can monitor the blockchains for the contract transactions. When it sees one that it has the spending transaction for it can extract the pre-image and insert it into the input script of the claim transaction. It can do this for both sides of the swap. If the initiator never reveals the pre-image the service can broadcast the refund transactions to revert the state back to normal (only after the timelock has expired of course). 

Inserting the pre-image into and already signed transaction is only possible thanks to segregated witness as otherwise the pre-image would aalter the transaction id and the signature would be invalid. 

This is as mentioned earlier not entirely risk free, a rouge service could choose to not broadcast a claim transaction in time and thus letting one participant claim both sides of the swap. If the service came with a small fee the service would be encouraged to not cheat the users and losing out on future fee revenue. Using a third-party is very convenient but comes at a price, in the form of risk. I would only recommend using such a service if you rarely do atomic swaps or you are swapping tiny amounts.


\Section{Off-chain atomic swaps}
How off-chain atomic swaps are performed are a bit harder to understand and 
required a lot more work, and yet in my opinion they hold more potential for 
the future. The limitation that comes with having more than one node running 
still exists, but the cost that comes with broadcasting transactions on-chain 
is almost completely removed. 

In my experiments I did open a proper channel and performed an atomic swap. 
I did not however do it over the real lightning  network, mostly due to time 
constraints and it not being necessary to prove that it worked. Doing it over 
a proper lightning network would require additional tweaks to the protocol and 
would take significantly more time. In my simulation I did not perform any 
multi-hop transactions over nodes not involved in the swap, the fact that it
still would work over multiple hops is evident in the mechanisms of the swap.

\Subsection{Solution proposal}
Doing the swaps on a payment channel that is not part of the lightning network
like I did in my experiments works fine, the problem is that the two participants
have to have channels open that are dedicated to doing swaps between them and nobody
else. This is fine for a couple of implementations, like for exchanges or banks 
etc. It would not be very fitting as a general solution however. 

To make off-chain atomic swaps possible in a more general setting, like the main
lightning network, some changes have to be made to the protocol. 

\Subsubsection{BOLT and c-lightning}
All nodes participating in the lightning network follow a set of rules and 
protocols that are defined in the BOLT specification. My first thought were to
simply propose a few changes to the protocol to allow swaps, but there is a 
simpler solution that makes the implementation even more versatile. 

The writers of the BOLT specification has purposely left openings in the
flag and protocol structure to allow for easy extensions. This is what is known
as ''it's ok to be odd rule'', what that means is that any  message with
and even number should be supported by all nodes. Messages with odd numbers 
are interpreted as optional. If the node receives a message with an odd number 
that it does not recognize it will simply be ignored (unknown even messages 
should result in closing of the connection as the nodes most likely
have different version numbers).

There are also flags in the BOLT specification that nodes tell each other about
upon connection. A flag is a bit set to 1 or 0 in a bytestring, and they 
represent a supported feature. As before these
flags follow the ''it's ok to be odd rule''. Thus flags in even positions are 
features that must be supported, and flags in odd positions represent optional
features that this node support.

BOLT is designed to support more than bitcoin with the help of the chain\_hash 
variable. The chain hash is the unique identifier for the genesis block on
whatever blockchain you refer to in a specifc protocol message. If a node
receives a chain hash it does not recognize or does not support it will ignore
that message.

Another interesting aspect is the new plugin system for the node implementation: 
\textbf{c-lightning}. This plugin system allows for functionality additions that
does not need to be part of the main implementation code. 

\Subsubsection{A non-disruptive solution}
My proposed solution involves making a plugin for c-lightning (and any other 
node as soon as they add plugin support). The plugin add an additional 
feature flag in any open odd position. The flag represents support and potential
willingness to perform swaps across chains. 

One of the aspects i find most interesting with the off-chain atomic swaps is
the fact that the other nodes (in the case of a multi-hop path) does not need
to know that the transaction they are propagating is part of a swap. In other 
words they do not need to have the plugin themselves to be part of the atomic
swap. Only the two end nodes on each side of the swap, meaning the two 
participants in the swap needs to run special plugin software. 

The plugin's normal operation outside the feature flag addition will be handling 
of forwarding new htlc outputs. When it receives a pre-image that is not related 
to any atomic-swap it will be handled like any normal htlc request would. 
When it sees a htlc related to a known atomic swap however it will handle it as 
a special case.

For now let's assume that the swapping parties reached their agreement through 
a separate channel. The details they need to agree upon are how much to be 
exchanged and at what rate, and also they need to make sure that there is a 
viable path between them on sides of the network, both to know if the swap 
is possible at all, and to know what timelock to start on. Let's also 
assume that Alice is the one who initiates the exchange.

In a normal transaction Bob would be the one who generates the pre-image used 
in the transaction chain started by Alice. In the case of a swap this would not 
work however, as she will ultimately be the nreceiver on the other end
when she receives the other type of crypto-currency. Thus before the exchange
begins Alice will generate the pre-image and sends the hash of it to Bob.

When these variables are agreed upon Alice can send the bitcoins to Bob on the
bitcoin side of the network. When the transaction reaches Bob is where the plugin
kicks in. It keeps track of agrred upon swaps and what pre-image hash that they
will use. When the new transaction comes in with the hash they agreed upon earlier
the plugin will act differently than normal operations. It will recognize the hash
and match it to the details of the swap. It can then send the promised money on 
the litecoin side of the network to Alice, using the same pre-image hash and 
continuing on the same timelock. 

As can be seen the plugin is quite simple but still allows for some complex 
behaviour, and it is very non-disruptive. It requires no extra software or special
nodes running on their own network. 

\Subsection{Alternative solutions}
